%%%%%%%%%%%%%%%%%%%%%%%%%%%%%%%%%%%%%%%%%
% Friggeri Resume/CV
% XeLaTeX Template
% Version 1.1 (9/2/15)
%
% This template has been downloaded from:
% http://www.LaTeXTemplates.com
%
% Original author:
% Adrien Friggeri (adrien@friggeri.net)
% https://github.com/afriggeri/CV
%
% License:
% CC BY-NC-SA 3.0 (http://creativecommons.org/licenses/by-nc-sa/3.0/)
%
% Important notes:
% This template needs to be compiled with XeLaTeX and the bibliography, if used,
% needs to be compiled with biber rather than bibtex.
%
%%%%%%%%%%%%%%%%%%%%%%%%%%%%%%%%%%%%%%%%%

\documentclass[]{friggeri-cv} % Add 'print' as an option into the square bracket to remove colors from this template for printing

\begin{document}

\header{Samuel}{ Brandão}{Software Engineer} % Your name and current job title/field

%----------------------------------------------------------------------------------------
%	SIDEBAR SECTION
%----------------------------------------------------------------------------------------

\begin{aside} % In the aside, each new line forces a line break
\section{contact}
Via Ruggero 
Leoncavallo, 10
Milan, 20131 - Italy
~
+39 3894570300
~
\href{mailto:gb.samuel@gmail.com}{gb.samuel@gmail.com}
%\href{http://www.smith.com}{http://www.smith.com}
\href{it.linkedin.com/in/sambrandao}{linkedin.com/in/sambrandao}
%\href{http://github.com/khamusa}{github:khamusa}
\section{languages}
portuguese native
english fluency
italian fluency
\section{programming}
{\color{red} $\varheartsuit$} Python 3
{\color{red} $\varheartsuit$} Ruby (\& Rails)
PostgreSQL, MySQL
MongoDB
Git
C
JavaScript
Angular, JQuery
CSS3 \& HTML5
F\#
PHP \& Wordpress
Erlang
Scala (Basic)
Java
\section{concepts}
{\color{red} $\varheartsuit$} Functional Programming
{\color{red} $\varheartsuit$} Database Design
{\color{red} $\varheartsuit$} Software Design
{\color{red} $\varheartsuit$} MetaProgramming
\end{aside}

%----------------------------------------------------------------------------------------
%	EDUCATION SECTION
%----------------------------------------------------------------------------------------

\section{about me}
I am a passionate software engineer currently looking for opportunities to 
learn/work with distributed concurrent systems and functional programming. I 
started programming at the age of 11 and working professionaly at the age of 19. 
While being a self-taught programmer contributed strongly for my problem solving
skills, I decided to pursue a more conscious and comprehensive knowledge of the 
computing ecosystem, enrolling for a Bachelor Degree in 2011. 

\section{education}
\begin{entrylist}
%------------------------------------------------
\entry
{2011--2015}
{Bachelor {\normalfont of Computer Science}}
{Università degli Studi di Milano}
{Laurea in Informatica}
%------------------------------------------------
%\entry
%{2007--2008}
%{Bachelor {\normalfont of Business Studies}}
%{The University of California, Berkeley}
%{Specialization in Commerce}
%------------------------------------------------
\end{entrylist}

%----------------------------------------------------------------------------------------
%	WORK EXPERIENCE SECTION
%----------------------------------------------------------------------------------------

\section{experience}
\begin{entrylist}
%------------------------------------------------
\entry
{2004-Now}
{FREELANCE DEVELOPER}
{Brazil, Italy, Brazil}
{\emph{Developer} \\ I've been working as a Freelance Developer approximately 
since 2005, having worked with a coprehensive set of technologies. Most of my
skills were self-taught,  
}
%------------------------------------------------
\end{entrylist}

%----------------------------------------------------------------------------------------
%	PROJECT SECTION
%----------------------------------------------------------------------------------------

\section{particular projects}

\begin{entrylist}
%------------------------------------------------
\entry
{2012-2014}
{Miviro}
{http://www.miviro.com.br}
{\emph{Lead Developer} \\
Miviro is a social network/management tool for tour agencies, developed with 
Ruby on Rails, HTML5, CSS3 and plenty of Javascript/Ajax. I've implemented 
dozens of functionalities, redesigned the overall app and database. 
Particularly interesting was the reporting system which allowed for 
administrators to retrieve summarized information using natural 
language queries as an interface to the database.
}
%------------------------------------------------
\entry
{2011}
{Wertix CMS}
{Standalone project}
{\emph{Lead Developer} \\
Wertix was a Content Management System developed from scratch as a mean to 
practice and learn the Ruby on Rails framework. It employed a very interesting 
concept of page blocks. Every content type created could be either displayed in 
standalone mode (a gallery, for example) or as a page block, which could then be
easily dragged-and-dropped to any position in any other page. 
The complexity of the system was on handling the polymorphic characteristics 
content types using relational databases (PostgreSQL).
}
%------------------------------------------------
\end{entrylist}

%----------------------------------------------------------------------------------------
%	COMMUNICATION SKILLS SECTION
%----------------------------------------------------------------------------------------

\section{communication skills}

\begin{entrylist}
%------------------------------------------------
\entry
{2005-2012}
{Bachelor in Drama}
{Universidade Federal de Minas Gerais}
{Acted as a professional actor/performed for about 6 years.}
%------------------------------------------------
\end{entrylist}

%----------------------------------------------------------------------------------------
%	INTERESTS SECTION
%----------------------------------------------------------------------------------------

\section{interests}
%\textbf{professional:} 
programming languages and paradigms (specially functional),
distributed systems, reactive platforms, actor models, concurrency, 
meta-programming. 
%----------------------------------------------------------------------------------------

\end{document}
